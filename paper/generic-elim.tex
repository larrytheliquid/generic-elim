\documentclass[preprint,authoryear]{sigplanconf}

\usepackage{todonotes}
\usepackage{amsmath}
\usepackage{url}

\begin{document}

\special{papersize=8.5in,11in}
\setlength{\pdfpageheight}{\paperheight}
\setlength{\pdfpagewidth}{\paperwidth}

\titlebanner{DRAFT}        % These are ignored unless
%% \preprintfooter{short description of paper}   % 'preprint' option specified.

\title{Generic Eliminators via Descriptions}
%% \subtitle{Curry Strikes Back}

\authorinfo{Larry Diehl\and Tim Sheard}
           {Portland State University}

\maketitle

\begin{abstract}
Sed pellentesque massa purus, ac aliquam ipsum auctor in. Mauris et
libero risus. In scelerisque neque vel odio tempus, commodo gravida
purus faucibus. Aliquam pharetra mauris consequat, gravida justo sed,
facilisis elit. Nullam enim ipsum, varius ut condimentum eget, tempor
id eros. Aenean id eros vestibulum, hendrerit dui eget, tincidunt
elit. Ut nec dignissim lectus, a bibendum turpis. Vivamus tincidunt
consequat ante. Suspendisse congue convallis ipsum vitae egestas. Ut
interdum elit id nunc aliquam lobortis.
\end{abstract}

\category{D.3}{Software}{Programming Languages}.

\keywords
Generic programming; dependent types; descriptions; eliminators.

\section{Introduction}
\label{sec:intro}

Descriptions make datatype definitions first class values in a
dependent type theory. This has several desirable consequences, such as the
ability to perform generic programming~[levitation, dagand thesis, ornaments]
over described types, as well as decreasing the number of constructs in the
metatheory via levitation. Any type represented as a description can
be eliminated with {\tt ind}, which resembles a fold (or catamorphism) but
over indexed type families. Definitions written with {\tt ind} are
verbose but follow a recognizable pattern (Section ~\ref{TODO}). 
On the other hand, definitions written with {\it eliminators} (or
induction principles) are shorter and more familiar to type theorists.

For example, the following eliminator of vectors has familiar
components. It consists of the motive argument and branches for each
vector constructor.

Additionally, many existing algorithms already translate to
eliminators. For example, to prove termination of Coq's fix or
dependent pattern matching. Rather than adapting such algorithms to
translate to {\tt ind}, we instead provide a generic eliminator {\tt elim} for
any datatype encoded by a description. Furthermore, {\tt elim} can be
defined in terms of {\tt ind}, making it another instance of generic
programming internal to the existing type theory.
Furthermore, a user writing a definition with {\tt ind} must be familiar
with the particular way we have chosen to encode a `Desc`ription as a
type. In contrast, users already familiar with eliminators need not
understand how descriptions work to be able to program with types
encoded with them [could contrast desc with Sum and desc with Σ Fin
here]. 

\bibliographystyle{abbrvnat}
\bibliography{generic-elim}

\end{document}
