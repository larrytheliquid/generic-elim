\documentclass[preprint,nonatbib]{sigplanconf}
\usepackage[authoryear,square]{natbib}

\usepackage{todonotes}
\usepackage{amsmath}
\usepackage{amsthm}
\usepackage{amssymb}
\usepackage{graphicx}
\usepackage{comment}
\usepackage{url}
\usepackage{bbm}
\usepackage[greek,english]{babel}
\usepackage{ucs}
\usepackage[utf8x]{inputenc}
\usepackage{autofe}
\usepackage{stmaryrd}
\usepackage{enumitem}

\usepackage{float}
\floatstyle{boxed}
\restylefloat{figure}

\DeclareUnicodeCharacter{8988}{\ensuremath{\ulcorner}}
\DeclareUnicodeCharacter{8989}{\ensuremath{\urcorner}}
\DeclareUnicodeCharacter{8803}{\ensuremath{\overline{\equiv}}}
\DeclareUnicodeCharacter{8759}{\ensuremath{\colon\colon}}
\DeclareUnicodeCharacter{12314}{\ensuremath{\llbracket}}
\DeclareUnicodeCharacter{12315}{\ensuremath{\rrbracket}}
\DeclareUnicodeCharacter{10214}{\ensuremath{\llbracket}}
\DeclareUnicodeCharacter{10215}{\ensuremath{\rrbracket}}
\DeclareUnicodeCharacter{8614}{\ensuremath{\mapsto}}
\DeclareUnicodeCharacter{8799}{\ensuremath{\stackrel{?}{=}}}
\DeclareUnicodeCharacter{8669}{\ensuremath{\leadsto}}

\DeclareUnicodeCharacter{7496}{\ensuremath{^{d}}}

\usepackage{fancyvrb}

\usepackage[labelfont=bf]{caption}

\newtheorem{mydef}{Definition}
\newtheorem{myparte}{Part$_E$}
\newtheorem{myparti}{Part$_I$}

\newcommand{\reffig}[1]{Figure \ref{fig:#1}}
\newcommand{\refsec}[1]{Section \ref{sec:#1}}
\newcommand{\refparte}[1]{Part$_E$ \ref{parte:#1}}
\newcommand{\refparti}[1]{Part$_I$ \ref{parti:#1}}

\begin{document}

\special{papersize=8.5in,11in}
\setlength{\pdfpageheight}{\paperheight}
\setlength{\pdfpagewidth}{\paperwidth}

\titlebanner{DRAFT}        % These are ignored unless
%% \preprintfooter{short description of paper}   % 'preprint' option specified.

\title{Generic Constructors and Eliminators from Descriptions}
\subtitle{Dependently typed programming without the algebra}

\authorinfo{Larry Diehl\and Tim Sheard}
           {Portland State University}

\maketitle

\begin{abstract}
Sed pellentesque massa purus, ac aliquam ipsum auctor in. Mauris et
libero risus. In scelerisque neque vel odio tempus, commodo gravida
purus faucibus. Aliquam pharetra mauris consequat, gravida justo sed,
facilisis elit. Nullam enim ipsum, varius ut condimentum eget, tempor
id eros. Aenean id eros vestibulum, hendrerit dui eget, tincidunt
elit. Ut nec dignissim lectus, a bibendum turpis. Vivamus tincidunt
consequat ante. Suspendisse congue convallis ipsum vitae egestas. Ut
interdum elit id nunc aliquam lobortis.
\end{abstract}

\category{D.3}{Software}{Programming Languages}.

\keywords
Generic programming; dependent types; descriptions; eliminators.

\section{Introduction}
\label{sec:intro}

Descriptions make datatype definitions first class values in a
dependent type theory. This has several desirable consequences, such as the
ability to perform generic programming~\citep{Chapman:2010:GAL:1932681.1863547,mcbride2010ornamental,dagand:phd}
over described types, as well as decreasing the number of constructs in the
metatheory via levitation~\citep{Chapman:2010:GAL:1932681.1863547,dagand:phd}. 
Any type represented as a description can
be eliminated with its elimination rule -- called {\tt ind} -- which resembles a fold (or catamorphism) but
over indexed type families. Functions written with {\tt ind} are
verbose but follow a recognizable pattern (\refsec{ind}). 
On the other hand, functions written with standard {\it eliminators} (or
induction principles) are shorter and more widely understood.

Now let's look at the type of {\tt ind}. Think of it almost like a standard
eliminator that has been parameterized over any description {\tt D} of a type. The
motive {\tt P} is indexed over the type family {\tt μ I D i}, which is
the datatype that description {\tt D} represents.
Here {\tt I} is the type of the index of the type family being
encoded, and {\tt i} is the value of the index whose type is {\tt I}.
The key difference
between {\tt ind} and a standard eliminator is that all branches for
each constructor of the encoded type are packaged into a single branch
{\tt α}. In order to prove that {\tt P} holds for any
{\tt μ I D i}, you get all constructors and all of their
arguments in {\tt xs}, along with all inductive
hypotheses {\tt ihs} for any recursive constructor arguments.

\newpage

\begin{verbatim}
ind : {I : Set} (D : Desc I)
  (P : (i : I) → μ D i → Set)
  (α : Hyps D X P ⇒ P ∘ init)
  (i : I) (x : μ D i) → P i x
\end{verbatim}

Function definitions written with {\tt ind} are verbose
because they need to unpack all of the data supplied by {\tt α}.
Worse still, there are several ways to define the datatype of
descriptions {\tt Desc}, and several ways to encode datatypes once a
particular definition of {\tt Desc} has been chosen. This means that a
user wishing to program with datatypes encoded by descriptions must be
aware of how the encodings get interpreted so that they may
appropriately unpack data when writing functions with {\tt ind}.

For example, if we partially apply {\tt ind} to a description of a
vector of booleans {\tt ind ℕ (VecD Bool)} then the type of the
resulting expression depends both on how we defined {\tt Desc} and how
we encoded vectors as a description {\tt VecD}.

Rather than making the user learn our data format and suffer the
consequences of unpacking encoded data, our {\bf primary contribution} is
the definition of a {\it generic} eliminator {\tt elim}!
It assumes a particular way of encoding
datatypes via descriptions, and partial application of such a
description to {\tt elim} results in the expected type of eliminator.
For example, partial application of our previous example to
{\tt elim} results in the following type.

\begin{verbatim}
elim ℕ (VecD Bool) :
  (P : (n : ℕ) → Vec Bool n → Set)
  (pnil : P zero nil)
  (pcons : (n : ℕ) (x : Bool) (xs : Vec Bool n)
    → P n xs → P (suc n) (cons n x xs))
  (n : ℕ) (xs : Vec Bool n) → P n xs
\end{verbatim}

Our {\tt elim} function can be defined in terms of
{\tt ind}, making it another example of generic programming made
possible by descriptions.
Note that we are not generating specialized eliminators like
{\tt elimVec} à la {\sc Coq}, but instead have a single generic
{\tt elim} function that can be applied to any description of a
datatype.
Because this definition is internal to the
existing type theory, no additional metatheory needs to be proven.
Furthermore, there are several existing algorithms already defined in
terms of eliminators that can now be reused.
For example, \citet{gimenez1995codifying} translates
{\sc Coq}~\citeyearpar{coq08} {\tt Fix}-based definitions into eliminators,
and \citet{Goguen06eliminatingdependent} and
\citet{McBride:2000:EM:646540.759262}
translate dependent pattern matching into
eliminators. Rather than reinventing these algorithms to translate to
{\tt ind}, we can instead directly reuse them with our derived {\tt elim}.

\section{Declaring Datatypes}
\label{sec:background}

The goal of this section is to {\it review} how to define the following type
declaration as a first-class value of our type theory. 

\begin{verbatim}
data Vec (A : Set) : ℕ → Set where
  nil : Vec A zero
  cons : (n : ℕ) (a : A) (xs : Vec A n)
    → Vec A (suc n)
\end{verbatim}

Whereas such a declaration typically involves axiomatically extending
the type theory, the technology of
descriptions~\citep{Chapman:2010:GAL:1932681.1863547,mcbride2010ornamental,dagand:phd}
lets us define datatypes within a closed type theory.
There are
several ways to define the datatype of descriptions {\tt Desc}. 
For simplicity, in this paper we use the encoding by
\citet{mcbride2010ornamental}.

\todo[inline]{cite dagand paper for actually translating data to desc}

\subsection{The Type of Descriptions}

The datatype {\tt Desc} of descriptions is used to represent
user-defined definitions of strictly-positive indexed
families of inductively defined types.
{\tt Desc} is paramterized by 
a type {\tt I}, the index of the encoded type family.

Throughout this paper it will be easier to first pretend like we
defined {\tt Vec} with a single constructor, either
{\tt nil} or {\tt cons}. This makes it easier to understand
later definitions where vector contains both constructors.

Imagine declaring a datatype with a single constructor.
A constructor is a sequence of
arguments that subsequent arguments may depend on, along with
recursive arguments at some type indices, and it ends with some type index.
Respectively, {\tt Arg}, {\tt Rec}, and {\tt End} allow you to encode
a dependent argument, a recursive argument at some index, and ending the
constructor definition at some index.

\begin{verbatim}
data Desc (I : Set) : Set₁ where
  End : (i : I) → Desc I
  Rec : (i : I) (D : Desc I) → Desc I
  Arg : (A : Set) (B : A → Desc I) → Desc I
\end{verbatim}

\subsection{Describing a Single Constructor}

For example, first recall the type of the
constructor {\tt nil} of vectors.

\begin{verbatim}
nil : (A : Set) → Vec A zero
\end{verbatim}

The constructor {\tt nil} takes no arguments, so its description
ends immediately at index {\tt zero}. The type of the description
returned is {\tt Desc ℕ} because the type we are encoding {\tt Vec}
is indexed by natural numbers.

\begin{verbatim}
nilD : (A : Set) → Desc ℕ
nilD A = End zero
\end{verbatim}

Next recall the type of the
constructor {\tt cons} of vectors.

\begin{verbatim}
cons : (A : Set) (n : ℕ) → A → Vec A n
  → Vec A (suc n)
\end{verbatim}

The description of {\tt cons} requires a dependent argument
{\tt n : ℕ} for the index, a non-dependent argument {\tt A} for the value
being added to the vector, a recursive argument indexed by the
natural number {\tt n}, and finally ends at index {\tt suc n}.

\begin{verbatim}
consD : (A : Set) → Desc ℕ
consD A =
  Arg ℕ (λ n → Arg A (λ _ → Rec n (End (suc n))))
\end{verbatim}

\subsection{Describing Multiple Constructors}
\label{sec:background:multiple}

The datatype {\tt Desc} can also be used to describe an entire
datatype, consisting of descriptions of multiple constructors.
This is achieved by making use of the isomorphism between disjoint
sums and dependent pairs whose domain is some finite enumeration.

\begin{verbatim}
A ⊎ B ≅ Σ Bool (λ b → if b then A else B)
\end{verbatim}

A datatype with multiple constructors is represented by an
{\tt Arg} description whose first argument {\tt VecT} is a datatype of tags
-- one for each constructor -- and whose second argument {\tt VecC} is
a function that returns a description for each constructor tag. Note
that whereas we used {\tt Arg} for arguments of constructors before,
now we are using {\tt Arg} to represent the sum of all constructors.

\begin{verbatim}
data VecT : Set where
  nilT consT : VecT

VecC : (A : Set) → VecT → Desc ℕ
VecC A nilT = End zero
VecC A consT =
  Arg ℕ (λ n → Arg A (λ _ → Rec n (End (suc n))))

VecD : (A : Set) → Desc ℕ
VecD A = Arg VecT (VecC A)
\end{verbatim}

Notice that our previous description of the {\tt nil} constructor,
{\tt nilD A}, is equal to the tagged description
{\tt VecC A nilT}, and {\tt consD A}
is equal to the tagged description
{\tt VecC A consT}.

\subsection{First-class Enumerations \& Tags}
\label{sec:background:case}

When defining the description of vectors, we previously used a custom
tag type {\tt VecT} to name each constructor. Descriptions are
primarily meant as a construction for representing user-defined
datatypes in a dependent type theory with a closed universe of types.
To prevent the need to extend the type theory with new tag types
constantly, we can instead define first-class enumerations and tags.
Enumerations are just a list of labels. A tag is an index into an
enumeration, pointing at a specific label.

\begin{verbatim}
Label : Set
Label = String

Enum : Set
Enum = List Label

data Tag : Enum → Set where
  here : ∀{l E} → Tag (l ∷ E)
  there : ∀{l E} → Tag E → Tag (l ∷ E)
\end{verbatim}

Thus, the type of vector tags {\tt VecT} can be defined as
{\tt Tag} applied to the enumeration {\tt "nil" ∷ "cons" ∷ []}.
We can also define the {\tt VecT} constructors
{\tt nilT} and {\tt consT} by using
{\tt Tag} constructors to index into the enumeration of labels. The
constructors {\tt here} and {\tt there} are analogous to {\tt zero}
and {\tt suc}.

\begin{verbatim}
VecE : Enum
VecE = "nil" ∷ "cons" ∷ []

VecT : Set
VecT = Tag VecE

nilT : VecT
nilT = here

consT : VecT
consT = there here
\end{verbatim}

A tag can be eliminated with a {\tt case} construct (this is referred to
as {\tt switch} by \citet{Chapman:2010:GAL:1932681.1863547,dagand:phd}).
In addition to the tag being elimnated, the case construct is given a
list of branches.

\begin{verbatim}
case : {E : Enum} (P : Tag E → Set)
  (cs : Branches E P) (t : Tag E) → P t
case P (c , cs) here = c
case P (c , cs) (there t) =
  case (λ t → P (there t)) cs t
\end{verbatim}

There is a branch for each label in the enumeration, and the type of
each branch depends on the tag representing the position of the label
in the enumeration.

\begin{verbatim}
Branches : (E : Enum) (P : Tag E → Set) → Set
Branches [] P = ⊤
Branches (l ∷ E) P =
  P here × Branches E (λ t → P (there t))
\end{verbatim}

Now we can redefine {\tt VecC} with the {\tt case} eliminator instead
of by pattern matching. Note that a right-nested product of
{\tt Branches} always ends with the unit type {\tt ⊤}.

\begin{verbatim}
VecC : (A : Set) → VecT → Desc ℕ
VecC A = case (λ _ → Desc ℕ)
  ( End zero
  , Arg ℕ (λ n → Arg A (λ _ → Rec n (End (suc n))))
  , tt )
\end{verbatim}

\section{Introducing with Algebras}
\label{sec:init}

The goal of this section is to {\it review} how to use the primitive
introduction rule for datatypes built using descriptions to define the
constructors of {\tt Vec}.

\begin{verbatim}
nil : (A : Set) → Vec A zero
cons : (A : Set) (n : ℕ) → A → Vec A n
  → Vec A (suc n)
\end{verbatim}

In a system where the datatype declaration {\tt Vec} is an axiomatic
extension, the constructors {\tt cons} and {\tt nil} are defined for
us. When using descriptions to define {\tt Vec}, we can instead
introduce values of type {\tt Vec} using its initial algebra.

\subsection{The Type of Fixpoints}

A description is a first-class datatype declaration. To get back the
type encoded by the description, you apply it to the fixpoint type
constructor {\tt μ}. For example, below we define {\tt Vec} by
applying {\tt μ} to its description {\tt VecD}.

\begin{verbatim}
data μ {I : Set} (D : Desc I) (i : I) : Set where
  init : El D (μ D) i → μ D i

Vec : (A : Set) (n : ℕ) → Set
Vec A n = μ (VecD A) n
\end{verbatim}

\subsection{Interpreting a Single Constructor}
\label{sec:init:el}

To introduce values of type {\tt Vec}, we use the
{\tt init} constructor of {\tt μ}. The argument to {\tt init} is
{\tt El D (μ D) i}. Let's understand {\tt El} by first considering a
description of {\tt Vec} that only has the single constructor
{\tt nil} or {\tt cons}. If {\tt init} introduces
a value of a single constructor datatype, then its arguments must be
the constructor's arguments. Thus, think of
{\tt El} as a function that computes the type of the arguments of our
constructor. {\tt El} computes the arguments as a right-nested tuple,
where {\tt Arg} gets interpreted as a dependent pair argument,
{\tt Rec} becomes a non-dependent recursive type argument, and
{\tt End} ends the tuple by requiring a proof that the constructor has
the correct index.

\begin{verbatim}
ISet : Set → Set₁
ISet I = I → Set

El : {I : Set} (D : Desc I) → ISet I → ISet I
El (End j) X i = j ≡ i
El (Rec j D) X i = X j × El D X i
El (Arg A B) X i = Σ A (λ a → El (B a) X i)
\end{verbatim}

The {\tt nil} constructor of vectors has no arguments. Thus,
{\tt El} for {\tt nilD} will only require a proof that the index in
the type is equal to the vector length zero.

\begin{verbatim}
NilEl : (A : Set) (n : ℕ) → Set
NilEl A n = El (nilD A) (Vec A) n

NilEl A n ⇝ zero ≡ n
\end{verbatim}

The {\tt cons} constructor of vectors has an index argument, an
argument for the value being added to the vector, a recursive
argument, and finally requires a proof that the index in the type is
equal to the successor of the index argument.

\begin{verbatim}
ConsEl : (A : Set) (n : ℕ) → Set
ConsEl A n = El (consD A) (Vec A) n

ConsEl A n ⇝
  Σ ℕ (λ m → A × Vec A m × (suc m ≡ n))
\end{verbatim}

\subsection{Interpreting Multiple Constructors}

Recall that multiple constructors are represented as a tagged sum
using a dependent pair (\refsec{background:multiple}). Thus,
{\tt El} for {\tt VecD} will be the tagged sum requiring
{\it either} {\tt NilEl} or {\tt ConsEl}.

\begin{verbatim}
VecEl : (A : Set) (n : ℕ) → Set
VecEl A n = El (VecD A) (Vec A) n

VecEl A n ⇝ Σ VecT (case (λ _ → Set)
  (NilEl A n , ConsEl A n , tt))
\end{verbatim}
\todo[inline]{motive is wrong}

\subsection{Defining Constructors}
\label{sec:init:cons}

We are now ready define the constructors {\tt nil} and
{\tt cons} using the initial algebra {\tt init}, which is the goal of
this section.
We have already seen
{\tt VecEl}, the type of the argument to {\tt init} for vectors.
Thus a constructor is defined by applying {\tt init} to a tuple. The
first argument is the tag choosing a particular constructor. Next comes
the tuple of proper arguments for the constructor. The tuple
ends with a proof that the index has the correct value.

\begin{verbatim}
nil : (A : Set) → Vec A zero
nil A = init (nilT , refl)

cons : (A : Set) (n : ℕ tt) (x : A) (xs : Vec A n)
  → Vec A (suc n)
cons A n x xs = init (consT , n , x , xs , refl)
\end{verbatim}

\section{Generic Constructors}
\label{sec:inj}

The goal of this section is to {\it contribute} a novel generic
constructor for datatypes built from descriptions. This constructor
may be used to define {\tt nil} and {\tt cons} as follows.

\begin{verbatim}
nil : (A : Set) → Vec A zero
nil A = inj (VecD A) nilT

cons : (A : Set) (n : ℕ tt) (x : A) (xs : Vec A n)
  → Vec A (suc n)
cons A = inj (VecD A) consT
\end{verbatim}

The constructors {\tt nil} and {\tt cons} are manually
defined in \refsec{init} using the initial algebra
{\tt init} as a primitive. Now we will define a generic constructor
{\tt inj} that once and for all captures the pattern inherent in
definitions of constructors. Importantly, our generic constructor is
defined in terms of the existing primitives and does not extend the
metatheory.
This amounts to:

\begin{myparti}
\label{parti:one}
Currying constructor arguments.
\end{myparti}

\begin{myparti}
\label{parti:two}
Inserting an implicit proof that the constructor has the correct index.
\end{myparti}

Defining {\tt inj} may not seem impressive by itself, but it acts as
nice pedagogical step towards understanding how to define the generic
eliminator {\tt elim} in \refsec{elim}.

\subsection{Uncurried Algebra}

In order to implement \refparti{one}, we must first recognize the
initial algebra as an uncurried function. Recall the type of the
initial algebra {\tt init : El D (μ D) i → μ D i}. Rather than
focusing on the initial algebra, we can generalize the uncurried view
of this constructor by replacing {\tt μ D} with an arbitrary
type family {\tt X : I → Set}.

\begin{verbatim}
UncurriedEl : {I : Set}
  (D : Desc I) (X : ISet I) → Set
UncurriedEl D X = ∀ {i} → El D X i → X i
\end{verbatim}

Recognize {\tt UncurriedEl} as an uncurried function by thinking of
{\tt El D X i} as a product of $n$ arguments $A_1 × ... × A_n$, an
argument requiring a proof of correct indexing $(j≡i)$, and
{\tt X i} as the result type $Z$.
\[
A_1 × ... × A_n × (j ≡ i) → Z
\]

For example, the applying {\tt UncurriedEl} to the description of the
{\tt cons} constructor results in the following type.

\begin{verbatim}
UncurriedEl (consD A) (Vec A) ⇝
  ∀{n} → ConsEl A n → Vec A n
\end{verbatim}

\subsection{Curried Algebra}

Now let's define the curried version of the function. Recall that the
type {\tt El} is a product of arguments, and {\tt UncurriedEl} is a
function from that product to some other type family. In contrast,
{\tt CurriedEl} is one large right-nested definition of function
arguments.

\begin{verbatim}
CurriedEl : {I : Set}
  (D : Desc I) (X : ISet I) → Set
CurriedEl (End i) X = X i
CurriedEl (Rec i D) X = (x : X i) → CurriedEl D X
CurriedEl (Arg A B) X = (a : A) → CurriedEl (B a) X
\end{verbatim}

Recognize {\tt CurriedEl} as a curried function that demands
$n$ constructor arguments as function arguments
$A_1 → ... → A_n$, and has the result type $Z$.
\[
A_1 → ... → A_n → Z
\]

Signifcantly, {\tt CurriedEl} does not require a proof of correct
indexing $(j≡i)$. Thus, in addition to solving \refparti{one} by
currying arguments, {\tt CurriedEl} also solves \refparti{two} by
implicitly supplying the correctness proof. Compare this to the
alternative definition {\tt CurriedEl'} that explicitly requires the
correctness proof below. The extra proof can be seen in the
{\tt End} constructor case.

\begin{verbatim}
CurriedEl' : {I : Set}
  (D : Desc I) (X : ISet I) (i : I) → Set
CurriedEl' (End j) X i =
  j ≡ i → X i
CurriedEl' (Rec j D) X i =
  (x : X j) → CurriedEl' D X i
CurriedEl' (Arg A B) X i =
  (a : A) → CurriedEl' (B a) X i
\end{verbatim}

Finally, below is an example of applying {\tt CurriedEl} to the
description of the {\tt cons} constructor. Notice that all arguments
are curried, and a proof of index correctness is not demanded.

\begin{verbatim}
CurriedEl (consD A) (Vec A) ⇝
  (m : ℕ) → A → Vec A m → Vec A (suc m)
\end{verbatim}

\subsection{Curry Function}

All we need now is a curry function that takes an
{\tt UncurriedEl} and returns a {\tt CurriedEl}. The definition of
this function is unremarkable, but its type clearly explains its
intentions.

\begin{verbatim}
curryEl : {I : Set} (D : Desc I) (X : ISet I)
  → UncurriedEl D X → CurriedEl D X
curryEl (End i) X cn =
  cn refl
curryEl (Rec i D) X cn =
  λ x → curryEl D X (λ xs → cn (x , xs))
curryEl (Arg A B) X cn =
  λ a → curryEl (B a) X (λ xs → cn (a , xs))
\end{verbatim}

\subsection{The Generic Constructor}

The moment has arrived, with the help of our {\tt curryEl} function we
can easily define the generic constructor {\tt inj}.

\begin{verbatim}
inj : {I : Set} (D : Desc I) → CurriedEl D (μ D)
inj D = curryEl D (μ D) init
\end{verbatim}

Unlike previous functions, this one is specialized to datatypes
defined with {\tt μ} rather than arbitrary type families {\tt X}. This
is the function we set out to define at the beginning of this section.
Compared to values of some type introduced with {\tt init} (\refsec{init:cons}),
values introduced with {\tt inj} have curried arguments and do not
need to supply a proof {\tt refl} of correct indexing.

\section{Eliminating with Algebras}
\label{sec:ind}

The goal of this section is to {\it review} how to use the primitive
elimination rule for datatypes built using descriptions. We use
the vector concatenation function (which flattens a vector of
homogenously-sized vectors) as our example, defined below using the
specialized eliminator {\tt elimVec}.

\begin{verbatim}
concat : (A : Set) (m n : ℕ)
  (xss : Vec (Vec A m) n) → Vec A (mult n m)
concat A m = elimVec (Vec A m)
  (λ n xss → Vec A (mult n m))
  (nil A)
  (λ n xs xss ih → append A m xs (mult n m) ih)
\end{verbatim}

This section develops the definitions necessary to understand how to
write {\tt concat} by applying the primitive elimination rule for
described types to a suitable algebra.

\subsection{The Type of Primitive Induction}

The type of the primitive elimination rule for datatypes built from
descriptions is given below.
The algebra {\tt α} is the important argument, as it is the proof that
that some property {\tt P} holds for any value of a
described type.
Whereas an eliminator
has seperate branches for proofs about each constructor, {\tt ind}
requires a single algebra argument that proves {\tt P} for any
constructor.

\begin{verbatim}
ind : {I : Set} (D : Desc I)
  (P : (i : I) → μ D i → Set)
  (α : (i : I) (xs : El D (μ D) i)
    (ihs : Hyps D (μ D) P i xs) → P i (init xs))
  (i : I) (x : μ D i) → P i x
\end{verbatim}

In order to prove {\tt P i (init xs)} you get the following
arguments:

\begin{enumerate}
\item{{\tt (i : I)}} - The index of the type being eliminated.
\item{{\tt (xs : El D (μ D) i)}} - The constructors and arguments of the type
  being eliminated.
\item{{\tt (ihs : Hyps D (μ D) P i xs) → P i (init xs))}} - The inductive
  hypotheses for all constructors.
\end{enumerate}

\citet{mcbride2010ornamental} gives the definition of {\tt ind}, but
our work can be understood without knowing the definition.

\subsection{Inductive Hypotheses for a Single Constructor}

{\tt Hyps} is the type of inductive hypotheses for a described
datatype. Its definition closely follows the definition of
the interpretation function of descriptions {\tt El}
(\refsec{init:el}).

\begin{verbatim}
Hyps : {I : Set} (D : Desc I) (X : ISet I)
  (P : (i : I) → X i → Set)
  (i : I) (xs : El D X i) → Set
Hyps (End j) X P i q = ⊤
Hyps (Rec j D) X P i (x , xs) =
  P j x × Hyps D X P i xs
Hyps (Arg A B) X P i (a , b) = Hyps (B a) X P i b
\end{verbatim}

First, let's understand {\tt Hyps} by what it computes for the
description of a single constructor like {\tt nil} or {\tt cons}.
{\tt Hyps} ignores dependent arguments {\tt Arg} and moves on, looking
for recursive arguments. When finding a recursive argument {\tt Rec}, it
asks for the motive {\tt P} instantiated at the recursive argument.
Finally, the tree of inductive hypotheses is terminated by the unit
type {\tt ⊤} once the description ends in {\tt End}.

The {\tt nil} constructor of vectors neither has dependent nor
recursive arguments. Thus, {\tt Hyps} for {\tt nilD} is simply
the unit type. Recall that {\tt NilEl} is the type that
{\tt nil}'s description gets interpreted as. The definition of
{\tt nilE} and related types can be found in \refsec{init:el}.

\begin{verbatim}
NilHyps : (A : Set)
  (P : (n : ℕ) → Vec A n → Set)
  (n : ℕ) (xs : NilEl A n) → Set
NilHyps A P n xs = Hyps (nilD A) (Vec A) P n xs

NilHyps A P zero refl ⇝ ⊤
\end{verbatim}

On the other hand, the {\tt cons} constructor of vectors requires an
inductive hypothesis for its recursive argument.

\begin{verbatim}
ConsHyps : (A : Set)
  (P : (n : ℕ) → Vec A n → Set)
  (n : ℕ) (xs : ConsEl A n) → Set
ConsHyps A P n xs = Hyps (consD A) (Vec A) P n xs

ConsHyps A P (suc m) (m , x , xs , refl) ⇝
  P m x × ⊤
\end{verbatim}

\subsection{Inductive Hypotheses for Multiple Constructors}

Once again, multiple constructors are represented by a tagged
sum (\refsec{background:multiple}). {\tt Hyps} for
{\tt VecD} requires {\it either} the inductive hypotheses of
{\tt nil} or the inductive hypotheses of {\tt cons}, depending on
which constructor {\tt Hyps} is applied to.

\begin{verbatim}
VecHyps : (A : Set)
  (P : (n : ℕ) → Vec A n → Set)
  (n : ℕ) (xs : VecEl A n) → Set
VecHyps A P n xs = Hyps (VecD A) (Vec A) P n xs

VecHyps A P n (nilT  , xs) ⇝ NilHyps  A P n xs 
VecHyps A P n (consT , xs) ⇝ ConsHyps A P n xs 
\end{verbatim}

\subsection{Defining Vector Concatenation with an Algebra}

Now we shall define the vector concatenation by applying the primitive
elimination rule for described types to an algebra. Below
{\tt concat} is defined as {\tt ind} applied to the description of
vectors, then the goal type as the motive, and finally the algebra
{\tt concatα}.
Note that we define the return type of
{\tt concat} to be {\tt Concat}, allowing us to reuse the return type
in later definitions.

\begin{verbatim}
Concat : (A : Set) (m n : ℕ)
  (xss : Vec (Vec A m) n) → Set
Concat A m n xss = Vec A (mult n m)

concat : (A : Set) (m n : ℕ)
  (xss : Vec (Vec A m) n) → Concat A m n xss
concat A m = ind
  (VecD (Vec A m))
  (Concat A m)
  (concatα A m)
\end{verbatim}

The algebra that defines {\tt concat} takes as arguments the index
{\tt n}, the constructors {\tt xss}, and the inductive hypotheses
{\tt ihs}. Recall that the type of vector constructors
{\tt xss : VecEl (Vec A m) n} is a dependent pair. The domain of the
pair is a vector tag {\tt VecT}, and the codomain is the type of
arguments corresponding to the constructor represented by the tag. We
eliminate the tag using {\tt case} (\refsec{background:case}), and
then provide a branches for the {\tt nil} and {\tt cons} constructors.

All definitions in this subsection are defined {\it without} dependent
pattern matching to illustrate the exclusive use of our type theory's
primitives ({\tt ind}, {\tt proj₁}, {\tt case}, etc). After we case
analyze the constructor tag in the first projection of {\tt xss}, we
need the {\it dependent} second projection to reduce to the arguments
of the constructor. This can be done by employing the
{\it convoy pattern}~\citep{TODO}, in which the special motive
{\tt ConcatConvoy} is passed to {\tt case}.

\begin{verbatim}
concatα : (A : Set) (m n : ℕ)
  (xss : VecEl (Vec A m) n)
  (ihs : VecHyps (Vec A m) (Concat A m) n xss)
  → Vec A (mult n m)
concatα A m n xss = case (ConcatConvoy A m n)
  (nilBranch A m n , consBranch A m n , tt)
  (proj₁ xss)
  (proj₂ xss)
\end{verbatim}

Again, rather than eliminating the pair {\tt xss}, we eliminate the
tag in the first projection using {\tt case}. The motive supplied to
case thus takes the first projection as an argument. The motive then
asks for the type of the second projection (dependent on the argument
supplied to the motive) as the argument {\tt xss}, in addition to the
remaining argument {\tt ihs}, and then the motive ends with the goal type
{\tt Vec A (mult n m)}.

\begin{verbatim}
ConcatConvoy : (A : Set) (m n : ℕ)
  → VecT → Set
ConcatConvoy A m n t =
  (xss : El (VecC (Vec A m) t) (Vec (Vec A m)) n)
  (ihs : VecHyps (Vec A m) (Concat A m) n (t , xss))
  → Vec A (mult n m)
\end{verbatim}

The {\tt nil} branch within the algebra's case analysis receives as
arguments the index {\tt n}, the single argument {\tt q}, and a value
{\tt u} of type unit as the inductive hypothesis. The argument
{\tt q} is not a proper argument of the constructor, but instead the
proof {\tt n ≡ zero}, stating that the index {\tt n} is equal to
{\tt zero} for the {\tt nil} constructor. One might expect to simply
define the {\tt nil} branch of {\tt concat} to return {\tt nil A}.
However, the type of the goal is {\tt Vec A (mult n m)} while the type
of {\tt nil A} is {\tt Vec A zero}. We can get the type of the goal to
reduce to {\tt Vec A (mult zero m)}, and then to {\tt Vec A zero}, by
applying our proof that {\tt n ≡ zero} to the equality coercion
function {\tt subst}.

\begin{verbatim}
nilBranch : (A : Set) (m n : ℕ)
  (xss : NilEl (Vec A m) n)
  (ihs : NilHyps (Vec A m) (Concat A m) n xss)
  → Vec A (mult n m)
nilBranch A m n q u = subst
  (λ n → Vec A (mult n m))
  q (nil A)
\end{verbatim}

Finally, the {\tt cons} branch is defined in much the same way. Note
that in {\sc Agda} an identifier is treated as single name unless it
contains a space. Thus, the argument {\tt n',xs,xss,q} below is a
single variable whose name reminds us of the tuple of constructor
arguments that it contains. Because we do not have access to pattern
matching, we need to project out each argument. For legibility, we
bind the names of the arguments below using a {\tt let} statement.
Unlike {\tt nil}, {\tt cons} has proper arguments but its tuple also
ends with a proof -- the proof that {\tt n ≡ suc n'}. The inductive
hypothesis of {\tt concat} is contained in the first projection of the
{\tt ih,u} argument, and the second projection is again a value of
type unit.

\begin{verbatim}
consBranch : (A : Set) (m n : ℕ)
  (xss : ConsEl (Vec A m) n)
  (ihs : ConsHyps (Vec A m) (Concat A m) n xss)
  → Vec A (mult n m)
consBranch A m n n',xs,xss,q ih,u =
  let n' = proj₁ n',xs,xss,q
      xs = proj₁ (proj₂ n',xs,xss,q)
      q = proj₂ (proj₂ (proj₂ n',xs,xss,q))
      ih = proj₁ ih,u
  in subst
    (λ n → Vec A (mult n m))
    q (append A m xs (mult n' m) ih)
\end{verbatim}

Congratulations on making it through this section, you now know how to define dependently typed
functions using the primitive elimination rule {\tt ind}!
Getting such function definitions right was a grueling experience for
the authors, and interactive theorem proving doesn't help much when
dealing with types that are so heavily encoded. You can relax knowing
that the next section defines a generic standard eliminator that we
can use to program with described datatypes instead of this
algebra-based approach.

\section{Generic Eliminators}
\label{sec:elim}

The goal of this section is to {\it contribute} a novel generic
eliminator for datatypes built from descriptions. This eliminator can
be used to define {\tt concat} as follows.

\begin{verbatim}
concat : (A : Set) (m n : ℕ tt)
  (xss : Vec (Vec A m) n) → Vec A (mult n m)
concat A m = elim VecE (VecC (Vec A m))
  (λ n xss → Vec A (mult n m))
  (nil A)
  (λ n xs xss ih → append A m xs (mult n m) ih)
\end{verbatim}

After partially applying {\tt elim} to an enumeration of constructor
names, and a function from tags (indexing into each constructor name)
to descriptions for each constructor, the resulting type is precisely
the interface of standard eliminators in type theory! 

The function {\tt concat} is defined in \refsec{ind} by applying the
primitive elimination rule {\tt ind} to an algebra. However,
functions defined in such a manner are verbose. Instead, now we
can define functions using our generic eliminator that once again can
be defined in terms of existing primitives without extending the
metatheory. This amounts to:

\begin{myparte}
\label{parte:one}
Performing case analysis to break up constructors into branches.
\end{myparte}

\begin{myparte}
\label{parte:two}
Currying constructor arguments in branches.
\end{myparte}

\begin{myparte}
\label{parte:three}
Inserting an implicit proof in each branch that the constructor has the correct index.
\end{myparte}

\subsection{Uncurried Algebra}

In order to implement
\refparte{two} and \refparte{three} we must recognize the algebra
argument to {\tt ind} as an uncurried function.
Below we define {\tt UncurriedHyps} to be a generalized type synonym
for the type of the algebra argument {\tt α} to {\tt ind}, where we
replace the fixpoint
{\tt μ D} with an arbitary type family {\tt X : I → Set}. This is
analogous to the generalization {\tt UncurriedEl} of the initial
algebra type in \refsec{inj}. In fact, because we generalize
{\tt UncurriedHyps} to be defined over arbitrary {\tt X} rather than
fixpoint {\tt μ D}, we require the extra argument
{\tt cn : UncurriedEl D X}, which you can think of as generic
constructor of {\tt X}.

\begin{verbatim}
UncurriedHyps : {I : Set}
  (D : Desc I) (X : ISet I)
  (P : (i : I) → X i → Set)
  (cn : UncurriedEl D X)
  → Set
UncurriedHyps D X P cn = ∀ i →
  (xs : El D X i)
  (ihs : Hyps D X P i xs)
  → P i (cn xs)
\end{verbatim}

Recognize {\tt UncurriedHyps} as a kind of uncurried function
consisting of one regular argument (the index type) and two product
arguments (the constructors and inductive hypotheses). 
Think of
{\tt El D X i} as a product of $n$ arguments plus the proof of correct
indexing $A_1 × ... × A_n × (j≡i)$, {\tt Hyps D X P i xs} as a
product of $m$ inductive hypotheses plus unit $B_1 × ... × B_n × ⊤$,
and {\tt X i} as the result type $Z$.
\[
I → A_1 × ... × A_n × (j ≡ i) → B_1 × ... × B_m × ⊤ → Z
\]

For example, we can use {\tt UncurriedHyps} to define the type of
{\tt consBranch} from \refsec{ind}.

\begin{verbatim}
UncurriedHyps (consD (Vec A m))
  (Vec (Vec A m))
  (Concat A m)
  (λ xs → init (consT , xs))
⇝
  (n : ℕ) (xss : ConsEl (Vec A m) n)
  (ihs : ConsHyps (Vec A m) (Concat A m) n xss)
  → Vec A (mult n m)
\end{verbatim}


%% \section{Generic Eliminators}
%% \label{sec:elim}

%% First let's recall the type of the algebra argument for {\tt ind}.

%% \begin{verbatim}
%% ∀ i
%% (xs : El D (μ D) i)
%% (ihs : Hyps D (μ D) P i xs)
%% → P i (init xs))
%% \end{verbatim}

%% You can think of this like an uncurried function of type
%% {\tt (A × B) → (C × D) → X}. Here {\tt xs} corresponds to uncurried
%% argument {\tt (A × B)}, {\tt ihs} corresponds to {\tt C × D}, and 
%% {\tt P i (init xs)} corresponds to the return type {\tt X}.

%% We can implement a function like {\tt ind} that takes a curried
%% version of the algebra argument instead. The {\tt CurriedAlg} argument
%% (defined in \refsec{curry}) represents the curried function
%% {\tt A → B → C → D → X}.

%% \begin{verbatim}
%% indCurried : {I : Set} (D : Desc I)
%%   (P : (i : I) → μ D i → Set)
%%   (α : CurriedAlg D (μ D) P init)
%%   (i : I)
%%   (x : μ D i)
%%   → P i x
%% indCurried D P α i x =
%%   ind D P (uncurryAlg D (μ D) P init α) i x
%% \end{verbatim}

%% The {\tt indCurried} functions implements 3 of the 4 parts of our
%% pattern. It starts off the definition with {\tt ind}
%% (\refparte{one}) and it implicitly projects arguments and inductive
%% hypotheses by using a curried {\tt α} argument
%% (\refparte{three}). Although it isn't obvious here, we will see
%% in \refsec{curry} that {\tt CurriedAlg} also implicitly
%% coerces constructor indices (\refparte{four}).

%% \subsection{Uncurried {\tt elim}}

%% To implement \refparte{two} we need to break up a sum of
%% constructors into branches, one for each constructor. Recall that the
%% shape of datatypes built out of descrptions that we would like to
%% eliminate is a sum of products. To capture this, our new induction
%% principle is specialied to descriptions representing a sum of
%% products. This is achieved by parameterizing not by any description,
%% but by an {\tt E : Enum} and a function {\tt C} from tags of that enumeration to
%% descriptions representing the constructor choices. We can use these
%% pieces to build a description starting with {\tt Arg}, as seen in the
%% {\tt let} construct below. Finally, \refparte{two} is implemented
%% by performing {\tt case} analysis in the body of {\tt ind}.

%% \begin{verbatim}
%% elimUncurried : {I : Set} (E : Enum) (C : Tag E → Desc I)
%%   → let D = Arg (Tag E) C in
%%   (P : (i : I) → μ D i → Set)
%%   → UncurriedBranches E
%%   (λ t → CurriedAlg (C t) (μ D) P (λ xs → init (t , xs)))
%%   ((i : I) (x : μ D i) → P i x)
%% elimUncurried E C P cs i x =
%%   let D = Arg (Tag E) C in
%%   indCurried D P
%%     (case (λ t →
%%       CurriedAlg (C t) (μ D) P (λ xs → init (t , xs)))
%%       cs)
%%     i x
%% \end{verbatim}

%% Here, {\tt UncurriedBranches} is just a type synonym for a function
%% from {\tt Branches E P} to some result type {\tt X}.
%% It's definition can
%% be found in \refsec{curry}. 
%% Above {\tt P} is
%% the {\tt CurriedAlg} argument, and {\tt X} is
%% {\tt (i : I) (x : μ D i) → P i x}.

%% \subsection{Curried {\tt elim}}

%% Although we have technically implemented all four parts of our
%% pattern, the induction principle {\tt elimUncurried} isn't quite what
%% we want. The ``uncurried'' part of the name of the function refers to
%% the {\tt Branches} argument.
%% {\tt UncurriedBranches} is analogous to the function
%% {\tt A × B → X}. Just like we did in {\tt indCurried}, we can define
%% our final generic eliminator {\tt elim} by returning
%% {\tt CurriedBranches} instead of {\tt UncurriedBranches}.
%% \linebreak
%% Here {\tt CurriedBranches} is analogous to {\tt A → B → X}.

%% \begin{verbatim}
%%   elim : {I : Set} (E : Enum) (C : Tag E → Desc I)
%%     → let D = Arg (Tag E) C in
%%     (P : (i : I) → μ D i → Set)
%%     → CurriedBranches E
%%     (λ t → CurriedAlg (C t) (μ D) P
%%       (λ xs → init (t , xs)))
%%     ((i : I) (x : μ D i) → P i x)
%%   elim E C P = curryBranches (elimUncurried E C P)
%% \end{verbatim}

%% Why do we need to do this? Recall that eliminators take a sequence of
%% branch arguments, one for each constructor. The type of
%% {\tt elimUncurried} is almost like a generic eliminator, except it
%% takes all branch arguments as a tuple (because that's what {\tt case}
%% takes), rather than taking a sequence of curried arguments. Thus, by
%% currying these branches we arrive at our final desired definition of
%% {\tt elim}.

%% The interface to using {\tt elim} is very simple. You simply replace
%% specialized eliminators with {\tt elim} applied to the tag and cases
%% that define a description. For example, the eliminator-based definitions of
%% {\tt concat} and {\tt append} in \refsec{background:elim} can
%% be defined by replacing {\tt elimVec} with {\tt elim VecT VecC}.

%% Finally, we should point out that we gave separate definitions, each
%% building upon the previous, for pedagogical reaons. Of course, {\tt elim}
%% can be defined in one go by inlining the other definitions.

%% \section{Introducing with Algebras}
%% \label{sec:init}

%% The opposite of eliminating values is introducing values.
%% A value of a type built from a description is introduced using
%% the initial algebra constructor {\tt init} of {\tt μ}. 

%% \subsection{The Type of {\tt init}}

%% Recall that the type of {\tt init} is {\tt El D (μ D) i → μ D i}.
%% When eliminating a
%% datatype with {\tt ind} you get {\tt El D (μ D) i} as
%% an argument to implement the algebra. If you want to access an
%% argument of some constructor contained in {\tt El D (μ D) i}, you
%% must project it out of the product of arguments.
%% When introducing a value you also use an algebra, the initial
%% one! Thus, to construct a value you need to pass it a tuple of
%% all arguments.

%% \subsection{Examples using {\tt init}}

%% Below are the specialized constructors
%% {\tt nil} and {\tt cons} for vectors.

%% \begin{verbatim}
%% nil : (A : Set) → Vec A zero
%% nil A = init          -- # 1
%%   (here               -- # 2
%%   , refl)             -- # 4

%% cons : (n : ℕ) → A → Vec A n → Vec A (suc n)
%% cons A n x xs = init  -- # 1
%%   (there here         -- # 2
%%   , n , x , xs        -- # 3
%%   , refl)             -- # 4
%% \end{verbatim}

%% Definitions using the initial algebra {\tt init} are much simpler than
%% definitions using {\tt ind}. To see this, just notice how much shorter
%% the definition of {\tt nil} and {\tt cons} above are compared to the
%% definitions of {\tt concat} and {\tt append} in \reffig{ind:concat}
%% and \reffig{ind:append}. Nonetheless, definitions of constructors
%% still can be broken up into 4 parts. The numbers of these parts are
%% referenced in the example code above.

%% \begin{myparti}[Introducing with {\tt init}]
%% \label{parti:one}
%% A definition begins by applying {\tt init} to a tuple of arguments.
%% \end{myparti}

%% \begin{myparti}[Choosing a tagged constructor]
%% \label{parti:two}
%% The first argument of the tuple is the tag that indexes into the
%% enumeration of constructor names. This tag represents the constructor
%% choice and  determines the type of subsequent elements of the tuple.
%% \end{myparti}

%% \begin{myparti}[Providing a tuple of arguments]
%% \label{parti:three}
%% The middle part of the tuple consists of all proper arguments of the
%% constructor. Some constructors, such as {\tt nil}, may not have any
%% proper arguments.
%% \end{myparti}

%% \begin{myparti}[Proving the constructor index correct]
%% \label{parti:three}
%% The final argument of the tuple is a proof that index of the codomain
%% of the initial algebra matches the index specified in the return
%% type of the constructor. This proof is merely the {\tt refl}
%% constructor of the equality type. For example, this proof would fail
%% if {\tt zero} was specified as the return type of {\tt cons}.
%% \end{myparti}

%% \section{Defining Generic Constructors}
%% \label{sec:inj}

%% In \refsec{init} we looked at examples of constructors defined
%% with {\tt init}, and how each such definition can be broken up into 4
%% parts. Now we will derive a generic constructor {\tt inj} that
%% implements these 4 parts.

%% \subsection{Curried {\tt init}}

%% First let’s once more recall the type of the initial
%% algebra {\tt init}.

%% \begin{verbatim}
%% El D (μ D) i → μ D i
%% \end{verbatim}

%% You can think of this like an uncurried function of type
%% {\tt (A , B) → X}, where {\tt A × B} corresponds to
%% {\tt El D (μ D) i} and {\tt X} corresponds to
%% {\tt μ D i}.

\section{Related Work}
\label{sec:related-work}

\citet{dagand:phd} gives an alternative {\tt Desc} definition, which
allows datatypes to be defined as computations over their indices.
In this paper a sum of datatypes, representing multiple construtors,
is encoded as a dependent pair whose first index is a list of choices
of constructor names. An alternative way to encode datatypes is to
support sum types directly in descriptions and use those rather than
their isomorphic dependent pair equivalents. Foveran~\citep{foveran} is
an example of a language that encoded sum types directly.

\todo[inline]{cite generic/generic-algebras.pdf}

\acks

\todo[inline]{thank reviewers + nathan for terminology + mention nsf grant}

\bibliographystyle{abbrvnat}
\bibliography{generic-elim}

\end{document}
